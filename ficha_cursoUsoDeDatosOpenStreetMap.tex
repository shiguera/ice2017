% Curso Excel facultad de informática
\documentclass[10pt, oneside]{article}

\usepackage[utf8]{inputenc}
\usepackage[spanish]{babel}
\usepackage[T1]{fontenc}

\usepackage[top=2cm, bottom=2cm, left=2cm, right=1.5cm]{geometry}
\usepackage[usenames]{color}

\definecolor{Azul}{RGB}{0,0,255}
\definecolor{SlateBlue}{RGB}{106,90,205}

\hyphenation{}

\begin{document}
\sffamily

\hspace*{-1em}\Large{\textbf{\textcolor{Azul}{Descarga y utilización de datos de OpenStreetMap}}}

\vspace{2em}
\hspace*{-2em}\textbf{\textcolor{SlateBlue}{Introducción}}

OpenStreetMap es la mayor base de datos libre de información geográfica existente. Es la wikipedia de los mapas. La posibilidad de editar y acceder a información geográfica de cualquier lugar del mundo y la cantidad de herramientas de uso libre existentes hacen de OpenStreetMap una herramienta de primer orden para los interesados en el GIS, la sociología, la ingeniería, los transportes y otros.

Durante el curso se aprenderá como acceder a los datos contenidos en la base de datos de OpenStreetMap y como utilizarlos a través de las herramientas GIS más habituales.

\vspace{1em}
\hspace*{-2em}\textbf{\textcolor{SlateBlue}{Objetivos}}
Los objetivos del curso son:

Entender cómo está estructurada la información geográfica en la base de datos de OpenStreetMap

Aprender a editar y descargar la información procedente de la base de datos de OpenStreetMap.

Aprender cómo utilizar las herramientas GIS que hay disponibles para manipular y utilizar dicha información.

\vspace{1em}
\hspace*{-2em}\textbf{\textcolor{SlateBlue}{Contenidos}}

\begin{enumerate}
\itemsep=0cm
  \item Estructura de la información en OpenStreetMap
  \item Edición con Id y JOSM
  \item Procedimientos para descargas de datos
  \item Utilización de QGIS con OpenStreetMap
  \item Conversión a formatos shapefile y de bases de datos
  \item Consultas selectivas con Overpass API
  \item Acceso mediante lenguajes de programación
\end{enumerate}

\vspace{1em}
\hspace*{-2em}\textbf{\textcolor{SlateBlue}{Metodología}}

Se proporcionará la documentación con los conceptos teóricos necesarios. Durante las clases se utilizarán dichos contenidos en la realización de ejercicios prácticos sobre herramientas GIS.

\hspace*{-2em}\textbf{\textcolor{SlateBlue}{Profesorado}}

Santiago Higuera de Frutos, Ingeniero de Caminos, Canales y Puertos.

Profesor asociado en el Departamento de Matemáticas e Informática aplicadas a la Ingeniería Civil y Naval de la UPM y miembro de la Fundación OpenStreetMap.

\vspace{1em}
\hspace*{-2em}\textbf{\textcolor{SlateBlue}{Datos de la actividad}}

\textbf{Duración: 8 horas} 

\textbf{Fechas:} 

\textbf{Horario:} 

\textbf{Lugar:} 

\end{document}
